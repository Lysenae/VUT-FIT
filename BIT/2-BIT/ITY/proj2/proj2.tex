\documentclass[a4paper,11pt,twocolumn]{article}
\usepackage[czech]{babel}
%\usepackage[latin2]{inputenc}
\usepackage[utf8]{inputenc}
\usepackage[left=1.25cm,text={18cm, 25cm},top=2.5cm]{geometry}
\usepackage{times}
\usepackage[IL2]{fontenc}
\usepackage{amsmath}
\usepackage{amsthm}
\usepackage{amsfonts}

\theoremstyle{definition}
\newtheorem{definition}{Definice}[section]
\newtheorem{theorem}{Veta}
\theoremstyle{plain}
\newtheorem{algorithm}[definition]{Algoritmus}


%ITY - Projekt č. 2 - Sazba dokumentů s matematickými výrazy
%-----------------------------------------------------------
%rozměry stránky: A4
%rozměry textové oblasti: 18x25cm
%mezera vlevo: 1.5cm
%mezera nahoře: 2.5cm
%font: times 11pt (vzorový dokument používá kódování fontů IL2)
%
%Poznámky: 
%- Na titulní straně uveďte své jméno 
%
%- Pomlčky a spojovníky jsou v tomto textu zadány znakem -. V dokumentu musí být správná šířka a mezery podle kontextu.
%
%- Vzorový dokument byl vysázen LaTeXem na školním serveru merlin těmito nástroji: latex, dvips -t a4, ps2pdf (pozor na křížové odkazy v textu, které jsou závislé na správném užití daných nástrojů pro sázení pdf)
%
%- Vzorce a některé další elementy jsou v tomto textu nahrazeny třemi tečkami. Řiďte se vzorovým dokumentem.
%
%- Z AmS balíků byly použity příkazy \verb|\newtheorem| a \verb|\binom|, plus další příkazy pro sázení speciálních symbolů.
%
%- Návěští definic, algoritmů, vět a důkazů, dále číslování nadpisů a vzorců, stejně jako vytváření odkazů na ně je potřeba sázet pomocí vhodných LaTeXových  prostředků, nikoli ručně.
%
%- Samozřejmě zde nesmí chybět správně vysázené prvky, které jste se naučili v 1. projektu.
%
%- Pozorně čtěte text v zadaném pdf, může vám být nápomocný k úspěšnému řešení projektu.
%
%- Odevzdává se pouze zdrojový text dokumentu a soubor Makefile pro překlad. 
%
%Text k vysázení
%---------------

\begin{document}

\begin{titlepage}

\begin{center}
\Huge
\textsc{Fakulta informačních technologií \\
Vysoké učení technické v Brně}
\vfill
\LARGE
Typografie a publikování - 2. projekt \\
Sazba dokumentů a matematických výrazů
\vfill
\end{center}
{\LARGE 2015 \hfill Daniel Klimaj}

\end{titlepage}

\section*{Úvod}
V této úloze si vyzkoušíme sazbu titulní strany, matematických vzorců, prostředí a dalších textových struktur obvyklých pro technicky zaměřené texty (například rovnice (\ref{eq1}) nebo definice \ref{def1} na straně \pageref{def1}).

Na titulní straně je využito sázení nadpisu podle optického středu s využitím zlatého řezu. Tento postup byl probírán na přednášce.


\section{Matematický text}
Nejprve se podíváme na sázení matematických symbolů a~výrazů v~plynulém textu. Pro množinu $V$~označuje card$(V)$ kardinalitu $V$.
Pro množinu~V reprezentuje $V^*$ volný monoid generovaný množinou $V$~s~operací konkatenace.
Prvek identity ve volném monoidu $V^*$ značíme symbolem $\varepsilon$.
Nechť $V^+ = V^* - \{\varepsilon\}$ Algebraicky je tedy $V^+$ volná pologrupa generovaná množinou $V$ s operací konkatenace.
Konečnou neprázdnou množinu $V$ nazvěme \emph{abeceda}.
Pro $w \in V^*$ označuje $|w|$ délku řetězce $w$. Pro $W \subseteq V$ označuje occur($w,W$) počet výskytů symbolů z~$W$ v~řetězci $w$ a~sym($w,i$) určuje $i$\--tý symbol řetězce $w$; například sym($abcd,3)=c$.

Nyní zkusíme sazbu definic a~vět s~využitím balíku \texttt{amsthm}.

\begin{definition}\label{def1}\emph{Bezkontextová gramatika} \normalfont je čtveřice $G=(V,T,P,S)$, kde $V$~je totální abeceda,
$T \subseteq V$ je abeceda terminálů, $S \in (V-T)$ je startující symbol a~$P$ je konečná množina pravidel
tvaru $q$:$\;A\rightarrow\alpha$, kde $A \in (V-T)$, $\alpha \in V^*$ a~$q$ je návěští tohoto pravidla. Nechť $N=V-T$ značí abecedu neterminálů.
Pokud $q:A\rightarrow\alpha\in P,\gamma,\delta\in V^*$, $G$ provádí derivační krok z~$\gamma A\delta$ do $\gamma\alpha\delta$ podle pravidla $q:A\rightarrow\alpha$, symbolicky píšeme 
$\gamma A\delta\Rightarrow\gamma\alpha\delta\;[q:A\rightarrow\alpha]$ nebo zjednodušeně $\gamma A\delta\Rightarrow\gamma\alpha\delta$ . Standardním způsobem definujeme $\Rightarrow^m$, kde $m\ge0$ . Dále definujeme 
tranzitivní uzávěr $\Rightarrow^+$ a~tranzitivně-reflexivní uzávěr $\Rightarrow^*$.\end{definition}

Algoritmus můžeme uvádět podobně jako definice textově, nebo využít pseudokódu vysázeného ve vhodném prostředí (například \texttt{algorithm2e}).

\begin{algorithm}Algoritmus pro ověření bezkontextovosti gramatiky. Mějme gramatiku $G=(N,T,P,S)$.
\begin{enumerate}
  \item Pro každé pravidlo $p\in P$ proveď test, zda $p$ na levé straně obsahuje právě jeden symbol z~$N$.
  \item Pokud všechna pravidla splňují podmínku z kroku 1, tak je gramatika $G$ bezkontextová.
\end{enumerate}
\end{algorithm}

\begin{definition}Jazyk definovaný gramatikou $G$ definujeme jako $L(G)={w\in T^* \;|\; S\Rightarrow^* w}$.\end{definition}

\subsection{Podsekce obsahující větu}

\begin{definition}Nechť $L$ je libovolný jazyk. $L$ je \emph{bezkontextový jazyk}, když a~jen když $L=L(G)$, kde $G$ je libovolná bezkontextová gramatika.\end{definition}

\begin{definition}Množinu $\mathcal{L}_{CF}= \{L|L$ nazýváme \emph{třídou bezkontextových jazyků}.\end{definition}

\begin{theorem}\label{thrm1}Nechť $L_{abc}=\{a^{n}b^{n}c^{n}|n\ge0\}$. Platí, že $L_{abc} \notin \mathcal{L}_{CF}$.\end{theorem}

\begin{proof}Důkaz se provede pomocí Pumping lemma pro bezkontextové jazyky, kdy ukážeme, že není možné, aby platilo, což bude implikovat pravdivost věty \ref{thrm1}.\end{proof}

\section{Rovnice a odkazy}

Složitější matematické formulace sázíme mimo plynulý text. Lze umístit několik výrazů na jeden řádek, ale pak je třeba tyto vhodně oddělit, například příkazem \verb|\quad|. 

$$\sqrt[x^2]{y_0^3}\quad \mathbb{N}=\{0,1,2,\ldots\}\quad x^{y^y} \neq x^{yy}\quad z_{i_j}\not\equiv z_{ij}$$\quad

V rovnici (\ref{eq1}) jsou využity tři typy závorek s různou explicitně definovanou velikostí.

\begin{equation}\label{eq1}
\bigg\{\bigg[(a+b)*c\bigg]^d+1\bigg\}=x
\end{equation}
\begin{equation*}
\lim \limits_{x \to \infty}\frac{\sin^2 x + \cos^2 x}{4}=y
\end{equation*}

V této větě vidíme, jak vypadá implicitní vysázení limity $\lim_{n \to \infty}f(n)$ v normálním odstavci textu. Podobně je to i s dalšími symboly jako $\sum_1^n$ či $\bigcup_{A\in \mathcal{B}}$. V případě vzorce $\lim \limits_{x \to 0}\frac{\sin x}{x}=1$ jsme si vynutili méně úspornou sazbu příkazem \verb|\limits|.

\begin{eqnarray}
&\int\limits_a^b f(x)dx = -\int_b^a f(x)dx& \\
&\left(\sqrt[5]{x^4}\right)' = \left(x^\frac{4}{5}\right)' = \frac{4}{5}x^{-\frac{1}{5}} = \frac{4}{5\sqrt[5]{x}}& \\
&\overline{\overline{{A}\lor{B}}} = \overline{\overline{A}\land\overline{B}}&\qquad
\end{eqnarray}

\section{Matice}

Pro sázení matic se velmi často používá prostředí array a závorky (\verb|\left|, \verb|\right|). 

$$ \left( \begin{array}{cc}
  a~+ b & b~- a \\
  \widehat{\xi+\omega} & \hat{\pi} \\
  \vec{a} & \overleftrightarrow{AC} \\
  0 & {\beta} \\
\end{array} \right) $$
$$ \mathbf{A} = \left \| \begin{array}{cccc}
   a_{11} & a_{12} & \ldots & a_{1n} \\
   a_{21} & a_{22} & \ldots & a_{2n} \\
   \vdots & \vdots & \ddots & \vdots \\
   a_{m1} & a_{m2} & \ldots & a_{mn} \\
 \end{array} \right \| $$
$$ \left| \begin{array}{ll}
  t~& u\\
  v & w \end{array} \right | = tw - uv
$$\quad

Prostředí array lze úspěšně využít i jinde.

$$\binom{n}{k} = \left\{ \begin{array}{ll}
  \frac{n!}{k!(n-k)!} & \mbox{pro $0 \le k \le n$}  \\
  0 & \mbox{pro $k < 0$ nebo $k > n$} \end{array} \right. $$

\section{Závěrem}

V případě, že budete potřebovat vyjádřit matematickou konstrukci nebo symbol a nebude se Vám dařit jej nalézt v samotném \LaTeX u, doporučuji prostudovat možnosti balíku maker \AmS-\LaTeX .
Analogická poučka platí obecně pro jakoukoli konstrukci v \TeX u.

\end{document}
