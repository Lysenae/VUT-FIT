\documentclass[a4paper,11pt,twocolumn]{article}
\usepackage[czech]{babel}
\usepackage[latin2]{inputenc}
\usepackage[left=2cm,text={17cm, 24cm},top=2.5cm]{geometry}
\usepackage{times}
\usepackage[IL2]{fontenc}

\newcommand{\myuv}[1]{\quotedblbase #1\textquotedblleft}

\title{Typografie a publikování \\
1. projekt}
\author{Daniel Klimaj \\
xklima22@stud.fit.vutbr.cz}
\date{}
\begin{document}
\twocolumn{\maketitle}
\section{Hladká sazba}

Hladká sazba je sazba z jednoho stupnì, druhu a øezu pí­sma sázená na stanovenou ¹íøku odstavce. Skládá se z odstavcù, které obvykle zaèínají­ zará¾kou, ale mohou být sázeny i bez zará¾ky -- rozhodují­cí­ je celková grafická úprava. Odstavce jsou ukonèeny východovou øádkou. Vìty nesmìjí zaèínat èíslicí.

Barevné zvýraznìní­, podtrhávání­ slov èi rùzné velikosti písma vybraných slov se zde také nepou¾ívá. Hladká sazba je urèena pøedev¹ím pro del¹í­ texty, jako je napøí­klad beletrie. Poru¹ení­ konzistence sazby pùsobí v textu ru¹ivì a unavuje ètenáøùv zrak.

\section{Smí¹ená sazba}

Smí¹ená sazba má o nìco volnìj¹í­ pravidla, jak hladká sazba. Nejèastìji se klasická hladká sazba doplòuje o dal¹í øezy pí­sma pro zvýraznìní­ dùle¾itých pojmù. Existuje \myuv{pravidlo}:

\bigskip
\begingroup
\scshape Èí­m ví­ce druhù, øezù, velikostí, barev pí­sma a jiných efektù pou¾ijeme, tí­m profesionálnìji bude  dokument vypadat. Ètenáø tím bude v¾dy nad¹en!
\endgroup
\bigskip

Tí­mto pravidlem se \underline{nikdy} nesmí­te øí­dit. Pøíli¹ èasté zvýrazòování textových elementù  a zmìny \begingroup \huge V\LARGE E\Large L\large I\normalsize K\small O\footnotesize S\scriptsize T\tiny I \endgroup pí­sma \begingroup \Large jsou \endgroup \begingroup \huge známkou \endgroup \begingroup \Huge \textbf{amatérismu} \endgroup autora a pùsobí­ \emph{\textbf{velmi} ru¹ivì}. Dobøe navr¾ený dokument nemá obsahovat ví­ce ne¾ 4 øezy èi druhy pí­sma. \texttt{Dobøe navr¾ený dokument je decentní­, ne chaotický}.

Dùle¾itým znakem správnì vysázeného dokumentu je konzistentní pou¾í­vání­ rùzných druhù zvýraznìní­. To napøí­klad mù¾e znamenat, ¾e \textbf{tuèný øez} pí­sma bude vyhrazen pouze pro klíèová slova, \emph{sklonìný øez} pouze pro doposud neznámé pojmy a nebude se to míchat. Sklonìný øez nepùsobí­ tak ru¹ivì a pou¾ívá se èastìji. V \LaTeX u jej sází­me radìji pøí­kazem \verb|\emph{text}| ne¾ \verb|\textit{text}|.

Smí¹ená sazba se nejèastìji pou¾ívá pro sazbu vìdeckých èlánkù a technických zpráv. U del¹í­ch dokumentù vìdeckého èi technického charakteru je zvykem upozornit ètenáøe na význam rùzných typù zvýraznìní­ v úvodní­ kapitole.

\section{Èeské odli¹nosti}

Èeská sazba se oproti okolní­mu svìtu v nìkterých aspektech mí­rnì li¹í­. Jednou z odli¹ností je sazba uvozovek. Uvozovky se v èe¹tinì pou¾í­vají­ pøevá¾nì pro zobrazení­ pøí­mé øeèi. V men¹í­ míøe se pou¾í­vají­ také pro zvýraznìní­ pøezdí­vek a ironie. V èe¹tinì se pou¾í­vá tento typ \myuv{uvozovek} namísto anglických \textquotedblleft uvozovek\textquotedblright.

Ve smí¹ené sazbì se øez uvozovek øí­dí­ øezem první­ho uvozovaného slova. Pokud je uvozována celá vìta, sází­ se koncová teèka pøed uvozovku, pokud se uvozuje slovo nebo èást vìty, sází­ se teèka za uvozovku.

Druhou odli¹ností je pravidlo pro sázení­ koncù øádkù. V èeské sazbì by øádek nemìl konèit osamocenou jednopí­smennou pøedlo¾kou nebo spojkou (spojkou \myuv{a} konèit mù¾e pøi sazbì do 25 liter). Abychom LaTeXu zabránili v sázení­ osamocených pøedlo¾ek, vkládáme mezi pøedlo¾ku a slovo nezlomitelnou mezeru znakem \textasciitilde{} (vlnka, tilda). Pro automatické doplnìní vlnek slou¾í­ volnì ¹iøitelný program vlna od pana Ol¹áka\footnote{Viz ftp://math.feld.cvut.cz/pub/olsak/vlna/.}.

\end{document}
